\documentclass[12pt,oneside,a4paper]{article}
\usepackage[left=1.5in, right=1in, top=1in, bottom=1in]{geometry}
\usepackage{setspace}
\doublespacing
\pagenumbering{gobble}
\title{DDOS DETECTION AND MITIGATION USING MACHINE LEARNING}
\author{By \\ ARPIT RAMESH GAWANDE}
\date{}
\begin{document}
\renewcommand{\thepage}{\roman{page}}% Roman numerals for page counter
\setcounter{page}{2}% Start page number with 2
\makeatletter
{\centering THESIS ABSTRACT \par
\vspace{3mm}
\large\@title \par
\vspace{3mm}
\@author \par
\vspace{3mm}
Thesis Director: \par
Dr. Jean-Camille Birget \par}
\\~\\~\\~\\ %4 spaces
Distributed Denial of Service (DDoS) attacks are very common nowadays. It is evident that the current industry solutions, such as completely relying on the Internet Service Provider (ISP) or setting up a DDoS defense infrastructure, are not sufficient in detecting and mitigating DDoS attacks, hence consistent research is needed. In this thesis we first tried to understand how DDoS attacks happen, then we discussed a way to detect DDoS attacks using machine learning tools at the routers, instead of setting up a centralized analysis system. We have proposed a standard communication architecture which can be used across all the networking devices for mitigating DDoS attacks. We have also created a simulation program to demonstrate our detection technique.
\end{document}
